\documentclass{beamer}
\usepackage[utf8]{inputenc}
\usepackage[T1]{fontenc}
\usepackage[italian]{babel}

\usepackage{xcolor}
\usepackage{graphicx}
\usepackage{amsmath, amssymb, bm}
\usepackage{booktabs}

\definecolor{unired}{HTML}{9B0014}
\usecolortheme[named=unired]{structure}
\usetheme{Madrid}

% Blocchi normali
\setbeamercolor{block title}{fg=white,bg=unired}
\setbeamercolor{block body}{fg=black,bg=unired!10}

% Blocchi di avviso
\setbeamercolor{block title alerted}{fg=white,bg=red!70!black}
\setbeamercolor{block body alerted}{fg=black,bg=red!10}

% Blocchi di esempio
\setbeamercolor{block title example}{fg=white,bg=unired!70}
\setbeamercolor{block body example}{fg=black,bg=unired!10}


\title[Mindfulness e stress]{Mindfulness e marcatori fisiologici dello stress}
\author{Federica Belardinelli}
\institute{Università degli Studi di Padova}
\date{16 settembre 2025}

\begin{document}

% --- Copertina ---
\begin{frame}[plain]
  \titlepage
\end{frame}

% --- Contesto ---
\section{Contesto}
\begin{frame}{Stress e salute}
  \begin{itemize}
    \item Lo stress è un \textbf{fattore di rischio} psicologico e fisiologico.
    \item Psicologico: ansia e depressione; Fisiologico: \textbf{SNS} e \textbf{asse HPA} attivati.
    \item Marcatori: \textbf{cortisolo}, frequenza cardiaca, pressione arteriosa.
  \end{itemize}
\end{frame}

\begin{frame}{Mindfulness}
  \begin{block}{Definizione}
    Insieme di pratiche per sviluppare \textbf{attenzione consapevole e non giudicante} al momento presente.
  \end{block}
  \begin{itemize}
    \item \textbf{FA}: attenzione focalizzata (es.\ respiro).
    \item \textbf{OM}: monitoraggio aperto di pensieri/sensazioni senza giudizio.
    \item \textbf{AST}: trascendimento spontaneo dell'attività mentale.
  \end{itemize}
\end{frame}

\begin{frame}{Studio di riferimento e focus della tesi}
  \begin{itemize}
    \item Pascoe et al.\ (2017): meta-analisi di \textbf{45 RCT} su marcatori dello stress.
    \item Nella tesi: focus su \textbf{3 indicatori}:
      \begin{enumerate}
        \item Cortisolo
        \item Pressione sistolica (SBP)
        \item Pressione diastolica (DBP)
      \end{enumerate}
  \end{itemize}
\end{frame}

% --- Meta-analisi ---
\section{Meta-analisi}
\begin{frame}{Cos’è la meta-analisi}
  \begin{block}{Definizione}
    Introdotta da Glass: \textbf{``analisi delle analisi''}. Combina studi indipendenti $\Rightarrow$ stime più \textbf{precise e affidabili}.
  \end{block}
  \begin{itemize}
    \item Media pesata per la precisione ($w_i = \sigma_i^{-2}$).
    \item Aumento della potenza statistica.
  \end{itemize}
\end{frame}

\begin{frame}{Modelli: Effetti fissi (FE) e casuali (RE)}
  \begin{columns}[T]
    \begin{column}{0.5\textwidth}
      \textbf{Effetti fissi (FE)}
      \begin{align*}
        Y_i &= \beta + \varepsilon_i, \quad \varepsilon_i \sim \mathcal{N}(0,\sigma_i^2) \\
        \hat{\beta}_{\text{FE}} &= \frac{\sum_i w_i Y_i}{\sum_i w_i}, \quad w_i = \sigma_i^{-2}
      \end{align*}
      \vspace{-0.5em}
      Ipotesi: unica vera misura d'effetto.
    \end{column}
    \begin{column}{0.5\textwidth}
      \textbf{Effetti casuali (RE)}
      \begin{align*}
        Y_i &= \beta_i + \varepsilon_i, \quad \beta_i = \beta + \eta_i, \quad \eta_i \sim \mathcal{N}(0,\tau^2) \\
        Y_i &\sim \mathcal{N}(\beta,\, \sigma_i^2 + \tau^2)
      \end{align*}
      \vspace{-0.5em}
      Ipotesi: l'effetto varia tra studi ($\tau^2$).
    \end{column}
  \end{columns}
\end{frame}

% --- Eterogeneità ---
\section{Eterogeneità}
\begin{frame}{Misurare l'eterogeneità}
  \begin{block}{Test Q di Cochran}
    \[
      Q = \sum_i w_i \,(Y_i - \hat{\beta}_{\text{FE}})^2
    \]
  \end{block}
  \begin{columns}[T]
    \begin{column}{0.5\textwidth}
      \begin{block}{Indice $I^2$}
        \[
          I^2 = \max\!\left(0,\, \frac{Q-(k-1)}{Q}\right)\times 100\%
        \]
      \end{block}
    \end{column}
    \begin{column}{0.5\textwidth}
      \begin{block}{Indice $H^2$}
        \[
          H^2 = \frac{Q}{k-1}, \qquad H = \sqrt{H^2}
        \]
      \end{block}
    \end{column}
  \end{columns}
\end{frame}

% --- Metodi di stima ---
\section{Metodi}
\begin{frame}{Stimare $\beta$ e $\tau^2$}
  \begin{itemize}
    \item \textbf{DerSimonian \& Laird} (classico).
    \item \textbf{Hartung--Knapp}: varianza aggiustata, uso di $t$ $\Rightarrow$ IC più realistici con pochi studi.
    \item \textbf{ML} (Massima verosimiglianza):
      \begin{equation*}
        \ell(\beta,\tau^2) = -\tfrac{1}{2}\sum_i \Big[ \log(\sigma_i^2+\tau^2) + \frac{(Y_i-\beta)^2}{\sigma_i^2+\tau^2} \Big]
      \end{equation*}
    \item \textbf{REML} (Verosimiglianza ristretta): massimizza la verosimiglianza dei residui, correggendo il bias di $\tau^2$.
  \end{itemize}
\end{frame}

\begin{frame}{REML: forma della log-verosimiglianza ristretta}
  \begin{block}{Impostazione}
    Con $V=\mathrm{diag}(\sigma_i^2)+\tau^2 I$, $X$ matrice del modello e $\hat{\beta}=(X^\top V^{-1}X)^{-1}X^\top V^{-1}Y$:
  \end{block}
  \begin{equation*}
    \ell_R(\tau^2) = -\tfrac{1}{2}\Big[\log|V| + \log|X^\top V^{-1}X| + (Y-X\hat{\beta})^\top V^{-1}(Y-X\hat{\beta})\Big] + C
  \end{equation*}
  \begin{itemize}
    \item Migliora la stima di $\tau^2$ soprattutto con \textbf{pochi studi}.
  \end{itemize}
\end{frame}

\begin{frame}{Correzioni per piccoli campioni}
  \begin{itemize}
    \item \textbf{Skovgaard}: correzione di ordine superiore della radice del log-rapporto di verosimiglianza $\Rightarrow$ test/IC più calibrati.
    \item \textbf{Bartlett}: riscalatura del log-rapporto di verosimiglianza per migliorare la copertura degli IC.
  \end{itemize}
\end{frame}

% --- Meta-regressione ---
\section{Meta-regressione}
\begin{frame}{Modello di meta-regressione}
  \begin{block}{Specificazione (RE con moderatori)}
    \[
      Y_i = \beta_0 + \beta_1 X_{i1} + \dots + \beta_p X_{ip} + u_i + \varepsilon_i,\quad
      u_i \sim \mathcal{N}(0,\tau^2),\ \varepsilon_i \sim \mathcal{N}(0,\sigma_i^2)
    \]
  \end{block}
  \begin{itemize}
    \item Scopo: spiegare l’\textbf{eterogeneità} tramite \textbf{moderatori}.
    \item Nella tesi: moderatore = \textbf{tipologia di pratica} (FA, OM, AST).
  \end{itemize}
\end{frame}

\begin{frame}
  \centering \Large Prima parte — Fine
\end{frame}

\end{document}
